\documentclass[11pt, article]{memoir}
% pre\'ambulo

\usepackage{lmodern}
\usepackage[T1]{fontenc}
\usepackage[spanish,activeacute]{babel}
\usepackage{mathtools}
\usepackage{babel}
\usepackage{textcomp}
\usepackage[utf8]{inputenc}
\usepackage{vmargin}
\usepackage{amsfonts}

\setpapersize{A4}
\setmargins{2.5cm}             % margen izquierdo
{1.5cm}                        % margen superior
{16.5cm}                       % anchura del texto
{23.42cm}                      % altura del texto
{10pt}                         % altura de los encabezados
{1cm}                          % espacio entre el texto y los encabezados
{0pt}                          % altura del pie de página
{2cm}                          % espacio entre el texto y el pie de página

\title{Exámen Seminario de Álgebra }
\author{Jorge Vasquez Arriaga}
\date{}
\renewcommand{\baselinestretch}{1.5}

\begin{document}
\maketitle

1.\textit{p} Encuentra el desarrollo en fracciones continuas simples de \\

(a) $-\dfrac{26}{17}$\\

Acordamos que dada la fracción continua $[a_{0}, a_{1}, ..., a_{k}]$  para toda  $i\not=0$ $a_{i}>0$\\
por lo que tomamos \\
$\left\lfloor -\dfrac{26}{17} \right\rfloor$ $=-2$\\
entonces\\
$-\dfrac{26}{17}=-2+\dfrac{8}{17}$\\
por el álgoritmo de Euclides tenemos \\
$17=8(2)+1$ \\
$8=1(8)+0$\\
por lo que la fracción continua simple de $-\dfrac{26}{17}$ esta dada por $[-2,2,8]$\\
y efectivamente \\
$-2+\dfrac{1}{2+\dfrac{1}{8}}=-2+\dfrac{1}{\dfrac{17}{8}}=-2+\dfrac{8}{17}=-\dfrac{26}{17}$\\


(b) $\sqrt{11} -2$\\

Notemos que $\left\lfloor \sqrt{11} -2 \right\rfloor$ $=1$, tenemos $\alpha_{0}=\sqrt{11}-2$ y $a_{0}=\left\lfloor \alpha_{0} \right\rfloor=1$ \\
${\alpha_{1}}=\dfrac{1}{\{\alpha_{0}\}}=\dfrac{1}{\sqrt{11}-2-1}=\dfrac{1}{\sqrt{11}-3}=\dfrac{\sqrt{11}+3}{2}$ y $a_{1}=\left\lfloor \alpha_{1} \right\rfloor=3$ \\
${\alpha_{2}}=\dfrac{1}{\{\alpha_{1}\}}=\dfrac{1}{\dfrac{\sqrt{11}+3}{2}-3}=\dfrac{1}{\dfrac{\sqrt{11}-3}{2}}=\dfrac{2}{\sqrt{11}-3}=\dfrac{2(\sqrt{11}+3)}{2}=\sqrt{11}+3$  y $a_{2}=\left\lfloor \alpha_{2} \right\rfloor=6$  \\
${\alpha_{3}}=\dfrac{1}{\{\alpha_{2}\}}=\dfrac{1}{\sqrt{11}+3-6}=\dfrac{1}{\sqrt{11}-3}$ y $a_{3}=\left\lfloor \alpha_{3} \right\rfloor=3$\\
de lo anterior tenemos $\alpha_{3}=\alpha_{1}$\\
 por lo que la fracción continua de $\sqrt{11} -2$ esta dada por $[1,\overline{3,6}]$ 

2. Demuestra que \\
\begin{center}
$\dfrac{h_{n}}{h_{n-1}}=[a_{n},a_{n-1},... ,{a_{0}}]$\\
\end{center}

Dem: lo probaremos por inducción con respecto a $a_{i}$\\
Caso base n=0\\
tenemos que\\
$h_{-2}=0, h_{-1}=1 $ y $ h_{0}=a_{0}h_{-1}+h_{-2}=a_{0}$\\
entonces\\
$\dfrac{h_{0}}{h_{-1}}=h_{0}=a_{0}=[a_{0}]$\\
por lo que el caso base se cumple.\\
Supongamos que se cumple para $n=k-1$, $\dfrac{h_{k-1}}{h_{k-2}}=[a_{k-1},a_{k-2},... ,{a_{0}}]$\\
veamos que se cumple para $\dfrac{h_{k}}{h_{k-1}},$\\
$\dfrac{h_{k}}{h_{k-1}}=\dfrac{a_{k}h_{k-1}+h_{k-2}}{h_{k-1}}=a_{k}+\dfrac{h_{k-2}}{h_{k-1}}$\\
y por hipotésis de inducción \\
$a_{k}+\dfrac{h_{k-2}}{h_{k-1}}=a_{k}+ \dfrac{1}{\dfrac{h_{k-1}}{h_{k-2}}}=a_{k}+ \dfrac{1}{[a_{k-1},a_{k-2},... ,{a_{0}}]}= a_{k}+\dfrac{1}{a_{k-1}+\dfrac{1}{a_{k-2}+\dfrac{1}{...+\dfrac{1}{a_{0}}}}}=[a_{k},a_{k-1},... ,{a_{0}}]$\\
entonces\\
$\dfrac{h_{k}}{h_{k-1}}=[a_{k},a_{k-1},... ,{a_{0}}]$\\
lo que concluye la inducción\\
y por lo tanto\\
$\dfrac{h_{n}}{h_{n-1}}=[a_{n},a_{n-1},... ,{a_{0}}]$
\begin{flushright}
$\square$
\end{flushright}

3. Demuestra que \\
\begin{center}
$\dfrac{h_{n}}{k_{n}}=\dfrac{h_{n+1}-h_{n-1}}{k_{n+1}-k_{n-1}}$\\
\end{center}

Dem: Sabemos que\\
$h_{i}=a_{i}h_{i-1}+h_{i-2}$ \\
entonces\\
$h_{n+1}=a_{n+1}h_{n}+h_{n-1}$\\
también tenemos\\
$k_{i}=a_{i}k_{i-1}+k_{i-2}$\\
entonces\\
$k_{n+1}=a_{n+1}k_{n}+k_{n-1}$ \\
de lo anterior\\
$\dfrac{h_{n+1}-h_{n-1}}{k_{n+1}-k_{n-1}}=\dfrac{a_{n+1}h_{n}+h_{n-1}-h_{n-1}}{a_{n+1}k_{n}+k_{n-1}-k_{n-1}}=\dfrac{a_{n+1}h_{n}}{a_{n+1}k_{n}}=\dfrac{h_{n}}{k_{n}}$\\
por lo tanto\\
$\dfrac{h_{n}}{k_{n}}=\dfrac{h_{n+1}-h_{n-1}}{k_{n+1}-k_{n-1}}$\\
\begin{flushright}
$\square$
\end{flushright}

4. Demuestra que \\
\begin{center}
$\dfrac{1}{k_{n}^{2}(a_{n+1}+2)} < \left|x-\dfrac{h_{n}}{k_{n}}\right|< \dfrac{1}{k_{n}^{2}(a_{n+1})}$\\
\end{center}

Sabemos que dada la fracción continua\\
$[a_{0},a_{1},a_{2},...,a_{n-1},x_{n}]=\dfrac{x_{n}h_{n-1}+h_{n-2}}{x_{n}k_{n-1}+k_{n-2}}$ donde $x_{n} \in $ $\mathbb{R}$\\
probaremos ahora que $h_{i}k_{i-1}-h_{i-1}k_{i}=(-1)^{i-1}$\\
lo probaremos por inducción sobre $i$,tenemos\\
$h_{-1}k_{-2}-h_{-2}k_{-1}=1(1)+0(0)=1$\\
continuando con la inducción, supongamos cierto\\
$h_{i-1}k_{i-2}-h_{i-2}k_{i-1}=(-1)^{i-2}$\\
veamos que se cumple para\\
$h_{i}k_{i-1}-h_{i-1}k_{i}$,\\
$h_{i}k_{i-1}-h_{i-1}k_{i}=(a_{i}h_{i-1}+h_{i-2})k_{i-2}-h_{i-1}(a_{i}k_{i-1}+k_{i-2})=-(h_{i-1}k_{i-2}-h_{i-2}k_{i-1})$\\
y por hipotésis de inducción\\
$-(h_{i-1}k_{i-2}-h_{i-2}k_{i-1})=(-1)^{i-1}$\\
lo que concluye la inducción,\\
veamos que\\
$x-r_{i-1}=x-\dfrac{h_{i-1}}{k_{i-1}}=\dfrac{x_{n}h_{i-1}+h_{i-2}}{x_{n}k_{i-1}+k_{i-2}}-\dfrac{h_{i-1}}{k_{i-1}}=\dfrac{-(h_{i-1}k_{i-2}-h_{i-2}k_{i-1})}{k_{i-1}(x_{i}k_{i-1}+k_{i-2})}=\dfrac{(-1)^{i-1}}{k_{i-1}(x_{i}k_{i-1}+k_{i-2})}$\\
si hacemos $i=n+1$ tenemos\\
$\left|x-r_{n}\right|=\left|x-\dfrac{h_{n}}{k_{n}}\right|=\dfrac{1}{k_{n}(x_{n+1}k_{n}+k_{n-1})}$\\
notemos que\\
$\dfrac{1}{k_{n}(x_{n+1}k_{n}+k_{n-1})}<\dfrac{1}{k_{n}(a_{n+1}k_{n}+k_{n-1})}<\dfrac{1}{k_{n}(a_{n+1}k_{n})}=\dfrac{1}{k_{n}^{2}a_{n+1}}$\\
y también\\
$\dfrac{1}{k_{n}(x_{n+1}k_{n}+k_{n-1})} \geq \dfrac{1}{k_{n}(x_{n+1}k_{n}+k_{n})} = \dfrac{1}{k_{n}(k_{n})(x_{n+1}+1)}>\dfrac{1}{k_{n}^{2}(a_{n+1}+1+1)}=\dfrac{1}{k_{n}^{2}(a_{n+1}+2)}$\\
de lo anterior podemos concluir que\\
$\dfrac{1}{k_{n}^{2}(a_{n+1}+2)} < \left|x-\dfrac{h_{n}}{k_{n}}\right|< \dfrac{1}{k_{n}^{2}(a_{n+1})}$
\begin{flushright}
$\square$
\end{flushright}

5. Encuentra tres números racionales $\dfrac{a}{b}$ tales que\\
\begin{center}
$\left|\dfrac{1+\sqrt{3}}{2}-\dfrac{a}{b}\right|< \dfrac{1}{2b^{2}}$\\
\end{center}

Sabemos que los convergentes son las mejores aproximaciones, entonces calculemos las sucesiones $h_{i}$ y $k_{i}$\\
como $h_{i}=a_{i}h_{i-1}+h_{i-2}$ y $k_{i}=a_{i}h_{i-1}+h_{i-2}$ necesitamos calcular la fraccion continua de $\dfrac{1+\sqrt{3}}{2}$\\
Notemos que $\left\lfloor \dfrac{1+\sqrt{3}}{2} \right\rfloor$ $=1$\\
tenemos $\alpha_{0}=\dfrac{1+\sqrt{3}}{2}$ y $a_{0}=\left\lfloor \alpha_{0} \right\rfloor=1$\\
${\alpha_{1}}=\dfrac{1}{\{\alpha_{0}\}}=\dfrac{1}{\dfrac{1+\sqrt{3}}{2}-1}=\dfrac{1}{\dfrac{-1+\sqrt{3}}{2}}=\dfrac{2}{\sqrt{3}-1}=\dfrac{2(\sqrt{3}+1)}{2}=\sqrt{3}+1$ y $a_{1}=\left\lfloor \alpha_{1} \right\rfloor=2$\\
${\alpha_{2}}=\dfrac{1}{\{\alpha_{1}\}}=\dfrac{1}{\sqrt{3}+1-2}=\dfrac{1}{\sqrt{3}-1}=\dfrac{\sqrt{3}+1}{2}$ y  $a_{2}=\left\lfloor \alpha_{2} \right\rfloor=1$\\
de lo anterior tenemos $\alpha_{2}=\alpha_{0}$ por lo que la fracción continua de $\dfrac{1+\sqrt{3}}{2}$ esta dada por $[1,\overline{2,1}]$\\
lo anterior nos permite calcular los convergentes\\
$r_{0}=\dfrac{h_{0}}{k_{0}}=\dfrac{a_{0}}{1}=a_{0}=1$\\
$r_{1}=\dfrac{h_{1}}{k_{1}}=\dfrac{a_{1}h_{0}+h_{-1}}{a_{1}k_{0}+k_{-1}}=\dfrac{2(1)+1}{2(1)+0}=\dfrac{3}{2}$\\
$r_{2}=\dfrac{h_{2}}{k_{2}}=\dfrac{a_{2}h_{1}+h_{0}}{a_{2}k_{1}+k_{0}}=\dfrac{1(3)+1}{1(2)+1}=\dfrac{4}{3}$\\
$r_{3}=\dfrac{h_{3}}{k_{3}}=\dfrac{a_{3}h_{2}+h_{1}}{a_{3}k_{2}+k_{1}}=\dfrac{2(4)+3}{2(3)+2}=\dfrac{11}{8}$\\
$r_{4}=\dfrac{h_{4}}{k_{4}}=\dfrac{a_{4}h_{3}+h_{2}}{a_{4}k_{3}+k_{2}}=\dfrac{1(11)+4}{1(8)+3}=\dfrac{15}{11}$\\
con las respectivas cuentas podemos darnos cuenta que $r_{0}, r_{2}, r_{4}$ cumplen la desigualdad\\
efectivamente\\
$\left|\dfrac{1+\sqrt{3}}{2}-\dfrac{1}{1}\right|=0.36<\dfrac{1}{2}=\dfrac{1}{2(1)^{2}}$\\
$\left|\dfrac{1+\sqrt{3}}{2}-\dfrac{4}{3}\right|=0.0326...<0.0\overline{5}=\dfrac{1}{18}=\dfrac{1}{2(3)^{2}}$\\
$\left|\dfrac{1+\sqrt{3}}{2}-\dfrac{11}{8}\right|=2.38...$x$10^{-3}<4.13...$x$10^{-3}=\dfrac{1}{242}=\dfrac{1}{2(15)^{2}}$\\
por lo que las fracciones, $r_{0}=\dfrac{1}{1},$ $ r_{2}=\dfrac{4}{3}, $ $r_{4}=\dfrac{15}{11}$ cumplen la desigualdad \\



6. Demuestra que si $x=[2^{3^{0}},2^{3^{1}},2^{3^{2}},2^{3^{3}},...]$ entonces sus convergentes cumplen\\
\begin{center}
$\left|x-\dfrac{h_{n}}{k_{n}}\right|<\dfrac{1}{k_{n}^{3}}$\\
\end{center}

Lo probaremos por inducción con respecto a los convergentes,\\
caso base $n=0$\\
$\left|x-\dfrac{h_{0}}{k_{0}}\right|=\left|x-a_{0}\right|=\dfrac{1}{2^{3^{1}}+\dfrac{1}{2^{3^{2}}...}}<1=\dfrac{1}{k_{n}^{3}}$\\
supongamos cierto para $n=i$  
por uno de los ejercicios anteriores tenemos que $\left|x-r_{i}\right|=\left|x-\dfrac{h_{i}}{k_{i}}\right|=\dfrac{1}{k_{i}(x_{i+1}k_{i}+k_{i-1})}$\\
y se sigue que\\
$\dfrac{1}{k_{i}(x_{i+1}k_{i}+k_{i-1})}<\dfrac{1}{k_{i}^{3}}$\\
veamos que\\
$\dfrac{1}{k_{i}(x_{i+1}k_{i}+k_{i-1})}=\dfrac{1}{k_{i}^{2}(x_{i+1}+\dfrac{k_{i-1}}{k_{i}})}$\\
por lo que\\
$\dfrac{1}{k_{i}^{2}(x_{i+1}+\dfrac{k_{i-1}}{k_{i}})}<\dfrac{1}{k_{i}^{3}}$\\
y es equivalente\\
$\dfrac{1}{x_{i+1}+\dfrac{k_{i-1}}{k_{i}}}<\dfrac{1}{k_{i}}$\\
lo que pasa si y solo sí\\
$x_{i+1}+\dfrac{k_{i-1}}{k_{i}}>k_{i}$\\
paso inductivo, lo queremos probar para $n=i+1$\\
$i.e.$  $\left|x-\dfrac{h_{i+1}}{k_{i+1}}\right|<\dfrac{1}{k_{i+1}^{3}}$\\
y haciendo un análisis análogo al anterior, basta probar que\\
$x_{i+1+1}+\dfrac{k_{i-1+1}}{k_{i+1}}>k_{i+1}$\\
y es equivalente a probar que\\
$x_{i+2}>k_{i+1}-\dfrac{k_{i}}{k_{i+1}}$\\
veamos que\\
$k_{i+1}-\dfrac{k_{i}}{k_{i+1}}<k_{i+1}=a_{i+1}k_{i}+k_{i-1}<a_{i+1}k_{i}+k_{i}$\\
por hipótesis de inducción\\
$(a_{i+1}+1)(k_{i})<(a_{i+1}+1)(x_{i+1}+\dfrac{k_{i-1}}{k_{i}})$\\
entonces\\
$(a_{i+1}+1)(x_{i+1}+\dfrac{k_{i-1}}{k_{i}})<(a_{i+1}+1)(x_{i+1}+1)<(x_{i+1}+1)^{2}=\left(a_{i+1}+\dfrac{1}{a_{i+2}+\dfrac{1}{...}}+1\right)^{2}=\left(2^{3^{i}}+\dfrac{1}{2^{3^{i+1}}+\dfrac{1}{...}}+1\right)^{2}<2^{3^{i+1}}+\dfrac{1}{2^{3^{i+2}}+\dfrac{1}{...}}=x_{n+1}$\\
por lo tanto $x_{i+2}>k_{i+1}-\dfrac{k_{i}}{k_{i+1}}$\\
y a partir de las desigualdes y equivalencias correspondientes concluimos la inducción\\
y por lo tanto\\
$\left|x-\dfrac{h_{n}}{k_{n}}\right|<\dfrac{1}{k_{n}^{3}}$\\
\begin{flushright}
$\square$
\end{flushright}

\end{document}
